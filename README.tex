\usemodule[readme]

\setupinteraction
  [title=ruby-lisp README,
   author=Nikolai Weibull]

\useURL [RDoc]      [http://rdoc.sourceforge.net/]
\useURL [Download]  [http://rubyforge.org/project/showfiles.php?group_id=109]
\useURL [Website]   [http://ned.rubyforge.org/ruby-lisp/]

\defineplacement[center][left=\hfill,right=\hfill]

\starttext

\startbodymatter


\title{Ruby-lisp ReadMe}

\section{Introduction}

“So what’s ruby-lisp?”, you ask.  Well, it’s a module for implementing nice
Lisp features for use in Ruby.  Currently the only sub||module that exists is
\type{Lisp::Format}, which implements Lisp’s \type{(format)} function and its
formatting language.  For people not familiar with this language, it’s a very
expressive string formatting language, much like \type{sprintf()}’s, but much
more powerful.  It allows for iteration, case conversion, conditionals, and
much more.  If you’ve ever felt stupid while creating a string iteratively, to
print an array or some such, this is definitely for you!


\section{Authors}

Ruby-lisp was written by the following people:

\startitemize
  \item Nikolai Weibull $<$nikolai@bitwi.se$>$
\stopitemize


\section{Documentation}

There is extensive documentation available, both of the implementation and how
to use the formatting language and its constructs, embedded in the source using
RDoc.  To generate the documentation, simply run RDoc on \FileName{format.rb}:

\startshellsession
$ rdoc format.rb
\stopshellsession

RDoc can be found at \from[RDoc].


\section{Download}

The current release can be found at \from[Download].


\section{News}

\subsection{Version 0.3.0: Adding a \FileName{NEWS} file.}

\startitemize
  \item Added NEWS file, as it was forgotten in previous release.
\stopitemize

\subsection{Version 0.2.0: Adding project||related files and a setup method.}

\startitemize
  \item Added a lot of project files, such as AUTHORS, ChangeLog, and so on.
  
  \item Added setup.rb script so that installation was made a bit more accessible.
\stopitemize

\subsection{Version 0.1.0: Initial release.}

Nothing to report; initial release.


\section{Install}

Ruby-lisp uses Minero Aoki’s \FileName{setup.rb} script to perform
installation.  What follows is an adapted introduction of how it works.


\subsection{How to set things up quickly.}


The following set of commands should be enough to install ruby-lisp for most
users.

\startshellsession
$ ruby setup.rb config
$ ruby setup.rb setup
# ruby setup.rb install
\stopshellsession

Lines beginning with a hash (\type{#}) may require root permissions.


\subsection{More of the details.}

Usage of setup.rb is:

\startshellsession
$ ruby setup.rb <global options>
$ ruby setup.rb [<global options>] <task> [<task options>]
\stopshellsession

Here, \type{<global options>} are any of the following:


  \placecenter
    \starttable[|l|l|]
    \NC \bf Option                  \NC \bf Meaning                         \NC\AR
    \HL
    \NC \type{-q}, \type{--quiet}   \NC Suppress message outputs            \NC\AR
    \NC \type{--verbose}            \NC Output messages verbosely (default) \NC\AR
    \NC \type{-h}, \type{--help}    \NC Prints help and quit                \NC\AR
    \NC \type{-v}, \type{--version} \NC Prints version and quit             \NC\AR
    \NC \type{--copyright}          \NC Prints copyright and quit           \NC\AR
    \stoptable

A task can be one of the following:

  \placecenter
    \starttable[|l|l|]
    \NC \bf Task                    \NC \bf Meaning                         \NC\AR
    \HL
    \NC \type{config}               \NC Checks and saves configurations     \NC\AR
    \NC \type{show}                 \NC Prints current configurations       \NC\AR
    \NC \type{setup}                \NC Compiles ruby extentions            \NC\AR
    \NC \type{install}              \NC Installs files                      \NC\AR
    \NC \type{clean}                \NC Cleans created files                \NC\AR
    \NC \type{distclean}            \NC Cleans created files (thouroughly)  \NC\AR
    \stoptable

The \type{config} task takes the following options:

  \placecenter
    \starttabulate[|l|lp|]
    \NC \bf Option                      \NC \bf Meaning                                                             \NC\AR
    \HL
    \NC \type{--prefix=PATH}            \NC A prefix of the installing directory path                               \NC\AR
    \NC \type{--std-ruby=PATH}          \NC The directory for standard ruby libraries                               \NC\AR
    \NC \type{--site-ruby-common=PATH}  \NC The directory for version-independent non-standard ruby libraries       \NC\AR
    \NC \type{--site-ruby=PATH}         \NC The directory for non-standard ruby libraries                           \NC\AR
    \NC \type{--bin-dir=PATH}           \NC The directory for commands                                              \NC\AR
    \NC \type{--rb-dir=PATH}            \NC The directory for ruby scripts                                          \NC\AR
    \NC \type{--so-dir=PATH}            \NC The directory for ruby extentions                                       \NC\AR
    \NC \type{--data-dir=PATH}          \NC The directory for shared data                                           \NC\AR
    \NC \type{--ruby-path=PATH}         \NC Path to set to she||bang line                                           \NC\AR
    \NC \type{--ruby-prog=PATH}         \NC The ruby program using for installation                                 \NC\AR
    \NC \type{--make-prog=NAME}         \NC The make program to compile ruby extentions                             \NC\AR
    \NC \type{--without-ext}            \NC Forces to \FileName{setup.rb} never to compile/install ruby extentions  \NC\AR
    \NC \type{--rbconfig=PATH}          \NC Your \FileName{rbconfig.rb} to load                                     \NC\AR
    \stoptabulate

You can view the default values of these options by typing:

\startshellsession
$ ruby setup.rb --help
\stopshellsession

If there’s the directory named \type{packages}, You can also use these options:

  \placecenter
    \starttable[|l|l|]
    \NC \bf Option                            \NC \bf Meaning                             \NC\AR
    \HL
    \NC \type{--with=NAME,NAME,NAME,}\dots    \NC Packages that you want to install       \NC\AR
    \NC \type{--without=NAME,NAME,NAME,}\dots \NC Packages that you don’t want to install \NC\AR
    \stoptable

You can pass options for extconf.rb like this:

\startshellsession
$ ruby setup.rb config -- \
> --with-tklib=/usr/lib/libtk-ja.so.8.0
\stopshellsession

The \type{install} task takes the following options:

  \placecenter
    \starttable[|l|l|]
    \NC \bf Option            \NC \bf Meaning                                 \NC\AR
    \HL
    \NC \type{--no-harm}      \NC Prints what to do and done nothing really   \NC\AR
    \NC \type{--prefix=PATH}  \NC The prefix of the installing directory path \NC\AR
    \stoptable


\section{Copying}

This library is free software; you can redistribute it and|/|or modify it under
the terms of the \GNU\ Lesser General Public License as published by the Free
Software Foundation; either version 2.1 of the license, or (at your option) any
later version.

This library is distributed in the hope that it will be useful, but
{\em without any warranty}; without even the implied warranty of
{\em merchantability} or {\em fitness for a particular purpose}.  See the \GNU\
Lesser General Public License for more details.

The \GNU\ Lesser General Public License can be found at \from[LGPL].  It can
also be requested by writing to the Free Software Foundation.  I bare no
responsibility for {\em you} finding this license, as all it does is grant
{\em you} rights that {\em you} wouldn’t have without it.


\section{Project Web||site}

The project website for ruby-lisp is currently located at \from[Website].


\stopbodymatter

\stoptext
