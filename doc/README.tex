\usemodule[readme]

\setupinteraction
  [title=ruby-lisp README,
   author=Nikolai Weibull]

\useURL [RDoc]      [http://rdoc.sourceforge.net/]
\useURL [Download]  [http://rubyforge.org/project/showfiles.php?group_id=109]
\useURL [Website]   [http://ned.rubyforge.org/ruby-lisp/]

\defineplacement[center][left=\hfill,right=\hfill]

\starttext

\startbodymatter


\title{Ruby-lisp ReadMe}

\section{Introduction}

“So what’s ruby-lisp?”, you ask.  Well, it’s a module for implementing nice
Lisp features for use in Ruby.  Currently the only sub||module that exists is
\type{Lisp::Format}, which implements Lisp’s \type{(format)} function and its
formatting language.  For people not familiar with this language, it’s a very
expressive string formatting language, much like \type{sprintf()}’s, but much
more powerful.  It allows for iteration, case conversion, conditionals, and
much more.  If you’ve ever felt stupid while creating a string iteratively, to
print an array or some such, this is definitely for you!


\section{Authors}

Ruby-lisp was written by the following people:

\startitemize
  \item Nikolai Weibull $<$nikolai@bitwi.se$>$
\stopitemize


\section{Documentation}

There is extensive documentation available, both of the implementation and how
to use the formatting language and its constructs, embedded in the source using
RDoc.  To generate the documentation, simply run RDoc on \FileName{format.rb}:

\startshellsession
$ rdoc format.rb
\stopshellsession

RDoc can be found at \from[RDoc].


\section{Download}

The current release can be found at \from[Download].


\section{News}

\subsection{Version 0.3.0: Adding a \FileName{NEWS} file.}

\startitemize
  \item Added NEWS file, as it was forgotten in previous release.
\stopitemize

\subsection{Version 0.2.0: Adding project||related files and a setup method.}

\startitemize
  \item Added a lot of project files, such as AUTHORS, ChangeLog, and so on.
  
  \item Added setup.rb script so that installation was made a bit more accessible.
\stopitemize

\subsection{Version 0.1.0: Initial release.}

Nothing to report; initial release.


\section{Install}

Ruby-lisp uses Minero Aoki’s \FileName{setup.rb} script to perform
installation.  What follows is an adapted introduction of how it works.

\getbuffer[INSTALL-setup.rb]


\section{Copying}

\getbuffer[COPYING-LGPL]

\section{Project Web||site}

The project website for ruby-lisp is currently located at \from[Website].


\stopbodymatter

\stoptext
